\documentclass[14pt,aspectratio=169]{beamer}

% Assets
\usepackage[czech]{babel}				% Jazyk
%\usepackage[a-2u]{pdfx}					% Kopírování z pdfka
\usepackage{tikz}						% Schémata automatů
\usepackage[utf8]{inputenc}
\usepackage{textcomp}
\usepackage{hyperref}

\usefonttheme{serif}
\usepackage{lmodern}

\hypersetup{%
    pdfencoding=auto,
    pdfauthor={\insertauthor},
    pdftitle={\insertsubtitle}
}
\usepackage{csquotes}					% české uvozovky
\usepackage{enumerate}					% enumerate environment
\usepackage{indentfirst}
\usepackage{mathtools}
\usepackage{pifont}
\usepackage{soul}
\usepackage{xcolor}
\usepackage{graphicx}
\usepackage{amsmath}
\usepackage{emoji}
\usepackage{subfig}

\usepackage{listings}                   % Úryvky z kódu
% Beamer theme
\usetheme{Darmstadt}
\useoutertheme[subsection=false]{miniframes}
\definecolor{red}{rgb}{0.827, 0, 0}
\usecolortheme[named=red]{structure}
\setbeamertemplate{frame numbering}[fraction]
\setbeamertemplate{navigation symbols}{}
% Enumerate
%\setlist[enumerate]{topsep=0pt,itemsep=-1ex,partopsep=1ex,parsep=1ex,label=(\arabic*)}

\MakeOuterQuote{"}

% Colors
\definecolor{darkblue}{rgb}{.047,.047,.43}
\definecolor{darkgreen}{HTML}{0D7103}
\definecolor{lightgreen}{HTML}{68FF00}
\definecolor{darkred}{HTML}{AF0B0B}
\definecolor{lightred}{HTML}{FF5100}
\definecolor{orange}{HTML}{FFE000}

\newcommand{\markred}[1]{\textcolor{lightred}{#1}}
\newcommand{\markgreen}[1]{\textcolor{lightgreen}{#1}}
\newcommand{\markorange}[1]{\textcolor{orange}{#1}}

% Inline images
\newcommand{\inlineimgscale}{1.1}

% X and check mark
\newcommand{\cmark}{\ding{51}}
\newcommand{\xmark}{\ding{55}}

% Redefinions
\renewcommand{\implies}{\Rightarrow}
\renewcommand{\impliedby}{\Leftarrow}

% Title page
\title{Informační a komunikační technologie}
\subtitle{Funkce v C}
\author{David Weber}
\def\office{K13}
\def\email{weber3@spsejecna.cz}

\begin{document}

    % Itemize
    \setlist[itemize]{label=\textcolor{white}{\textbullet}}

    % Slides
    \begin{frame}
        \titlepage
    \end{frame}

    \begin{frame}[t,fragile]{Příklad na úvod}
        \begin{lstlisting}
// Print array
for (int i = 0; i < size; i++) {
    printf("%d ", arr[i]);
}
printf("\n");
// Square array elements
for (int i = 0; i < size; i++) {
    arr[i] *= arr[i];
}
// Print array
for (int i = 0; i < size; i++) {
    printf("%d ", arr[i]);
}
        \end{lstlisting}
    \end{frame}

    \begin{frame}[t,fragile]{V čem je problém?}
        \begin{itemize}
            \item Kód je funkční, ale část pro výpis pole je zde uvedena \textbf{dvakrát}.
            \begin{lstlisting}
for (int i = 0; i < size; i++) {
    printf("%d ", arr[i]);
}
            \end{lstlisting}
            \item \markred{Opakující kód je nepraktický} \xmark
            \begin{itemize}
                \item budeme-li chtít změnit nějakou jeho část, musíme změnu provést všude
            \end{itemize}
            \item \markgreen{Použijeme funkci} \cmark
        \end{itemize}
    \end{frame}

    \begin{frame}[t]{Co je to funkce?}
        \begin{itemize}
            \item Obecně se jedná o část programu, kterou je možné opakovaně ``vyvolat'' v různých místech programu.
            \item 
        \end{itemize}
    \end{frame}

    \begin{frame}[t,fragile]{Struktura funkce}
        \begin{lstlisting}
navratovy_typ jmeno(parametry) {
    <implementace>
}
        \end{lstlisting}
        \begin{itemize}
            \item U funkce je třeba specifikovat:
            \begin{itemize}
                \item návratový datový typ,
                \item jméno (identifikátor),
                \item parametry,
                \item tělo (implementace).
            \end{itemize}
            \item Funkce uvádíme mimo tělo funkce \texttt{main}.
        \end{itemize}
    \end{frame}

    \begin{frame}[t,fragile]{Funkce bez návratové hodnoty a parametrů}
        \begin{itemize}
            \item Nejjednodušší typ funkce.
            \item Klíčové slovo \markblue{\texttt{void}}, prázdné závorky \texttt{()}
            \begin{lstlisting}
void greet() {
    printf("Hello World!");
}
            \end{lstlisting}
            \item Samotná deklarace funkce nic nedělá $\implies$ \markorange{je třeba ji tzv. \textbf{zavolat}.}
        \end{itemize}
    \end{frame}

    \begin{frame}[t,fragile]{Volání funkce}
        \begin{itemize}
            \item Je třeba specifikovat, kde v programu se má daná funkce provést.
            \begin{lstlisting}
void greet() {
    printf("Hello World!");
}
            
int main() {

    greet();    // Prints out "Hello World!"

    return 0;
}
            \end{lstlisting}
        \end{itemize}
    \end{frame}

    \begin{frame}{Otázky?}
        \begin{figure}
            \centering
            \includegraphics[scale=.4]{images/discussion_inverted.png}
        \end{figure}
    \end{frame}

\end{document}