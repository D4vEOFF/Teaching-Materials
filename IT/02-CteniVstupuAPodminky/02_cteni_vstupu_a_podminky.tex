\documentclass[14pt]{beamer}

\usepackage[czech]{babel}				% Jazyk
%\usepackage[a-2u]{pdfx}					% Kopírování z pdfka
\usepackage{tikz}						% Schémata automatů
\usepackage[utf8]{inputenc}
\usepackage{textcomp}
\usepackage{hyperref}

\usefonttheme{serif}
\usepackage{lmodern}

\hypersetup{%
    pdfencoding=auto,
    pdfauthor={\insertauthor},
    pdftitle={\insertsubtitle}
}
\usepackage{csquotes}					% české uvozovky
\usepackage{enumerate}					% enumerate environment
\usepackage{indentfirst}
\usepackage{mathtools}
\usepackage{pifont}
\usepackage{soul}
\usepackage{xcolor}
\usepackage{graphicx}
\usepackage{amsmath}
\usepackage{emoji}
\usepackage{subfig}

\usepackage{listings}                   % Úryvky z kódu

% Assets
% Beamer theme
\usetheme{Darmstadt}
\useoutertheme[subsection=false]{miniframes}
\definecolor{red}{rgb}{0.827, 0, 0}
\usecolortheme[named=red]{structure}
\setbeamertemplate{frame numbering}[fraction]
\setbeamertemplate{navigation symbols}{}
% Enumerate
%\setlist[enumerate]{topsep=0pt,itemsep=-1ex,partopsep=1ex,parsep=1ex,label=(\arabic*)}

\MakeOuterQuote{"}

% Colors
\definecolor{darkblue}{rgb}{.047,.047,.43}
\definecolor{darkgreen}{HTML}{0D7103}
\definecolor{lightgreen}{HTML}{68FF00}
\definecolor{darkred}{HTML}{AF0B0B}
\definecolor{lightred}{HTML}{FF5100}
\definecolor{orange}{HTML}{FFE000}

\newcommand{\markred}[1]{\textcolor{lightred}{#1}}
\newcommand{\markgreen}[1]{\textcolor{lightgreen}{#1}}
\newcommand{\markorange}[1]{\textcolor{orange}{#1}}

% Inline images
\newcommand{\inlineimgscale}{1.1}

% X and check mark
\newcommand{\cmark}{\ding{51}}
\newcommand{\xmark}{\ding{55}}

% Redefinions
\renewcommand{\implies}{\Rightarrow}
\renewcommand{\impliedby}{\Leftarrow}

% Title page
\title{Informační a komunikační technologie}
\subtitle{První pohled na jazyk C}
\author{David Weber}
\def\office{K13}
\def\email{weber3@spsejecna.cz}

\begin{document}

    % Itemize
    \setlist[itemize]{label=\textcolor{white}{\textbullet}}

    % Slides
    \begin{frame}
        \titlepage
    \end{frame}

    \begin{frame}[t,fragile]{Co už známe\dots}
        \begin{itemize}
            \item Tělo programu a vkládání knihoven.
            \begin{lstlisting}
#include <stdio.h>

int main(void) {

    // Kód programu

    return 0;
}
            \end{lstlisting}
            \item \texttt{printf(\dots)} -- příkaz pro výpis textového řetězce.
        \end{itemize}
    \end{frame}

    \begin{frame}[t]{Co dnes probereme}
        \begin{itemize}
            \item Čtení vstupu od uživatele
            \item Výpis proměnné
            \item Větvení programu (podmínky)
        \end{itemize}
    \end{frame}

    \begin{frame}[t]{Čtení vstupu}
        \begin{itemize}
            \item Příkaz \texttt{scanf(\textbf{<formátovací řetězec>}, \&\textbf{<proměnná>});}
            \item Formátovací řetězec
            \begin{itemize}
                \item Specifikuje \markorange{datový typ} vstupní hodnoty
                \item Základní položky:
                \begin{itemize}
                    \item \texttt{\%d} $\leftarrow$ celé číslo (\markgreen{int})
                    \item \texttt{\%f} $\leftarrow$ číslo s desetinnou částí (\markorange{float})
                    \item \texttt{\%s} $\leftarrow$ řetězec znaků (\markorange{string})
                \end{itemize}
                \item Píšeme do uvozovek \textquotedbl{} \textquotedbl (je to řetězec).
            \end{itemize}
        \end{itemize}
    \end{frame}

    \begin{frame}[t,fragile]{Příklad}
        \begin{lstlisting}
#include <stdio.h>

int main() {
    int x;
    scanf("%d", &x);
    return 0;
}
        \end{lstlisting}
    \end{frame}

    \begin{frame}[t,fragile]{Výpis proměnné}
        \begin{itemize}
            \item Provádíme také pomocí formátovacího řetězce.
            \item Pomocí \%d\,/\,\%f\,/\,\%s specifikujeme typ proměnné, kterou vypisujeme.
            \item Už nepíšeme \&.
            \begin{lstlisting}
int x;
x = 5;
printf("%d", x);
            \end{lstlisting}
            \item Program vypíše 5.
        \end{itemize}
    \end{frame}

    \begin{frame}[t,fragile]{Více hodnot}
        \begin{itemize}
            \item V rámci jednoho příkazu lze načíst/vypsat i více hodnot.
            \item Příklad pro dvě hodnoty
            \begin{lstlisting}
int x;
int y;
scanf("%d %d", &x, &y);
printf("%d %d", x, y);
            \end{lstlisting}
            \item Lze načítat/vypisovat libovolný počet hodnot.
        \end{itemize}
    \end{frame}

    \begin{frame}[t,fragile]{Podmínky}
        \begin{itemize}
            \item Umožňují reagovat na různé situace, které mohou v programu nastat.
            \item Skládá se z \textbf{logického výrazu} určujícího podmínku a tzv. \textbf{těla}.
            \item Klíčové slovo \texttt{if}.
            \begin{lstlisting}
if (podmínka) {
    // Kód programu
}
            \end{lstlisting}
        \end{itemize}
    \end{frame}

    \begin{frame}[t,fragile]{Příklad}
        \begin{lstlisting}
int x;
scanf("%d", &x);
if (x > 5) {
    printf("x je vetsi nez 5");
}
        \end{lstlisting}
    \end{frame}

    \begin{frame}[t]{Operátory}
        \begin{itemize}
            \item Seznam operátorů:
            \begin{itemize}
                \item \texttt{==} (rovnost)
                \item \texttt{!=} (nerovnost)
                \item \texttt{<} (menší než)
                \item \texttt{>} (větší než)
                \item \texttt{<=} (menší než nebo rovno)
                \item \texttt{>=} (větší než nebo rovno)
            \end{itemize}
        \end{itemize}
    \end{frame}

    \begin{frame}[t]{Logické operátory}
        \begin{itemize}
            \item Umožňují "sloučení" více podmínek do jedné.
            \item Seznam operátorů:
            \begin{itemize}
                \item \texttt{\&\&} - logický \textbf{AND} (podmínka je splněna, pokud jsou \textbf{všechny} dílčí podmínky splněny).
                \item \texttt{||} - logický \textbf{OR} (podmínka je splněna, pokud je splněna \textbf{alespoň jedna} z dílčích podmínek).
            \end{itemize}
            \item Syntaxe:
            \begin{center}
                \texttt{if (podm$_1$ \&\& podm$_2$ \&\& \dots{ }\&\& podm$_n$)}\\
                \texttt{if (podm$_1$ || podm$_2$ || \dots{ }|| podm$_n$)}
            \end{center}
        \end{itemize}
    \end{frame}

    \begin{frame}[t,fragile]{Příklad}
        \begin{itemize}
            \item Podmínka, že zadané číslo $x$ je z intervalu $\langle 0,\,5\rangle$
            \begin{lstlisting}
#include <stdio.h>

int main(void) {
    int x;
    scanf("%d", &x);

    if (x >= 0 && x <= 5) {
        printf("Cislo x je z intervalu (0, 5).");
    }
}
            \end{lstlisting}
        \end{itemize}
    \end{frame}

    \begin{frame}{Otázky?}
        \begin{figure}
            \centering
            \includegraphics[scale=.4]{images/discussion_inverted.png}
        \end{figure}
    \end{frame}

\end{document}