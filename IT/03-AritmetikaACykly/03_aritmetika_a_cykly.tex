\documentclass[14pt,aspectratio=169]{beamer}

\usepackage[czech]{babel}				% Jazyk
%\usepackage[a-2u]{pdfx}					% Kopírování z pdfka
\usepackage{tikz}						% Schémata automatů
\usepackage[utf8]{inputenc}
\usepackage{textcomp}
\usepackage{hyperref}

\usefonttheme{serif}
\usepackage{lmodern}

\hypersetup{%
    pdfencoding=auto,
    pdfauthor={\insertauthor},
    pdftitle={\insertsubtitle}
}
\usepackage{csquotes}					% české uvozovky
\usepackage{enumerate}					% enumerate environment
\usepackage{indentfirst}
\usepackage{mathtools}
\usepackage{pifont}
\usepackage{soul}
\usepackage{xcolor}
\usepackage{graphicx}
\usepackage{amsmath}
\usepackage{emoji}
\usepackage{subfig}

\usepackage{listings}                   % Úryvky z kódu

% Assets
% Beamer theme
\usetheme{Darmstadt}
\useoutertheme[subsection=false]{miniframes}
\definecolor{red}{rgb}{0.827, 0, 0}
\usecolortheme[named=red]{structure}
\setbeamertemplate{frame numbering}[fraction]
\setbeamertemplate{navigation symbols}{}
% Enumerate
%\setlist[enumerate]{topsep=0pt,itemsep=-1ex,partopsep=1ex,parsep=1ex,label=(\arabic*)}

\MakeOuterQuote{"}

% Colors
\definecolor{darkblue}{rgb}{.047,.047,.43}
\definecolor{darkgreen}{HTML}{0D7103}
\definecolor{lightgreen}{HTML}{68FF00}
\definecolor{darkred}{HTML}{AF0B0B}
\definecolor{lightred}{HTML}{FF5100}
\definecolor{orange}{HTML}{FFE000}

\newcommand{\markred}[1]{\textcolor{lightred}{#1}}
\newcommand{\markgreen}[1]{\textcolor{lightgreen}{#1}}
\newcommand{\markorange}[1]{\textcolor{orange}{#1}}

% Inline images
\newcommand{\inlineimgscale}{1.1}

% X and check mark
\newcommand{\cmark}{\ding{51}}
\newcommand{\xmark}{\ding{55}}

% Redefinions
\renewcommand{\implies}{\Rightarrow}
\renewcommand{\impliedby}{\Leftarrow}

% Title page
\title{Informační a komunikační technologie}
\subtitle{Aritmetika a cykly v C}
\author{David Weber}
\def\office{K13}
\def\email{weber3@spsejecna.cz}

\begin{document}

    % Itemize
    \setlist[itemize]{label=\textcolor{white}{\textbullet}}

    % Slides
    \begin{frame}
        \titlepage
    \end{frame}

    \begin{frame}[t,fragile]{Připomenutí z minula}
        \begin{itemize}
            \item Příkazy:
            \begin{itemize}
                \item \texttt{printf(\dots)} $\rightarrow$ výpis
                \item \texttt{scanf(\dots)} $\rightarrow$ načtení vstupu
                \item \texttt{\%d, \%f, \%s, \dots} $\rightarrow$ formátové specifikace
            \end{itemize}
            \item Větvení programu:
            \begin{lstlisting}
if (podmínka) {
    // Kód programu
}
            \end{lstlisting}
            \item Logické spojky \texttt{\&\&} a \texttt{||}.
        \end{itemize}
    \end{frame}

    \begin{frame}[t]{Co dnes probereme}
        \begin{itemize}
            \item Datové typy
            \item Aritmetické operátory a knihovna \markorange{\texttt{math}}
            \item Cykly \markorange{\texttt{while}} a \markorange{\texttt{for}}
        \end{itemize}
    \end{frame}

    \begin{frame}[t]{Datové typy}
        \begin{itemize}
            \item \textbf{Datový typ} definuje druh hodnot, kterých smí nabývat proměnná (nebo konstanta).
            \item 2 druhy:
            \begin{itemize}
                \item \textbf{jednoduché} $\leftarrow$ typicky přímou součástí jazyka,
                \item \textbf{složené} $\leftarrow$ obsahují více prvků (stejného či různých datových typů).
            \end{itemize}
            \item Deklarace v jazyce C: \texttt{\markblue{<datový typ>} <název proměnné>}.
        \end{itemize}
    \end{frame}

    \begin{frame}[t]{S čím budeme pracovat\dots}
        \begin{itemize}
            \item \markblue{int} (integer) $\leftarrow$ celočíselný datový typ
            \begin{itemize}
                \item $16$ nebo $32$ bitů (typicky $32$)
            \end{itemize}
            \item \markblue{float} (floating point) $\leftarrow$ desetinný datový typ
            \begin{itemize}
                \item $32$ bitů
                \item uchovává $6$ platných cifer.
            \end{itemize}
            \item \markblue{double} $\leftarrow$ desetinný datový typ (dvojnásobný rozsah oproti \texttt{float})
            \begin{itemize}
                \item $64$ bitů
                \item uchovává $15$ platných cifer.
            \end{itemize}
            \item Přehled všech datových typů v C (pro zájemce): \href{https://en.wikipedia.org/wiki/C_data_types}{\underline{zde}}
        \end{itemize}
    \end{frame}

    \begin{frame}[t]{Další datové typy}
        \begin{itemize}
            \item \markblue{char} $\leftarrow$ reprezentuje číslo nebo znak
            \begin{itemize}
                \item 8 bitů
            \end{itemize}
            \item Řetězce znaků \texttt{\markorange{\textquotedblleft\dots\textquotedblright}} $\leftarrow$ složitější, zatím nebudeme řešit \emoji{slightly-smiling-face}
            \item \markblue{void} $\leftarrow$ prázdný datový typ, nenabývá žádné hodnoty.
        \end{itemize}
    \end{frame}

    \begin{frame}[t]{Aritmetika}
        \begin{itemize}
            \item Jazyk C v základu podporuje standardní aritmetické operace.
            \item Operátory:
            \begin{itemize}
                \item Sčítání \texttt{+}
                \item Odčítání \texttt{-}
                \item Násobení \texttt{*}
                \item Dělení \texttt{/}
            \end{itemize}
            Priorita operací je stejná jako v matematice, popř. ji lze stanovit pomocí závorek \texttt{(,)}. 
        \end{itemize}
    \end{frame}

    \begin{frame}[t,fragile]{Příklad}
        \begin{lstlisting}
#include <stdio.h>

int main(void) {
    int a, b;
    scanf("%d %d", &a, &b);

    int sum = a + b;

    printf("%d", sum);
}
        \end{lstlisting}
    \end{frame}

    \begin{frame}[t,fragile]{Příklad}
        Procentuální podíl $x$ vůči $y$:
        \begin{lstlisting}
#include <stdio.h>

int main(void) {
    float x, y;
    scanf("%f %f", &x, &y);
    
    float percent = 100 * x / y;
    
    printf("%g", percent);

    return 0;
}
        \end{lstlisting}
    \end{frame}

    \begin{frame}[t]{Knihovna \texttt{math}}
        \begin{itemize}
            \item Standardní aritmetické operace nám již nebudou stačit, pokud budeme chtít počítat se složitějšími výrazy.
            \item Knihovna \texttt{math} poskytuje různé matematické funkce pro počítání např. s
            \begin{itemize}
                \item \textbf{mocninami},
                \item \textbf{odmocninami},
                \item \textbf{goniometrickými funkcemi},
                \item \textbf{logaritmy},
                \item \dots
            \end{itemize}
        \end{itemize}
    \end{frame}

    \begin{frame}[t]{Přehled některých funkcí}
        \begin{itemize}
            \item Co se vám může hodit:
            \begin{itemize}
                \item \texttt{pow(x, n)} $\leftarrow$ vypočítá $x^n$.
                \item \texttt{sqrt(x)} $\leftarrow$ vypočítá $\sqrt{x}$.
                \item \texttt{floor(x)} $\leftarrow$ zaokrouhlení čísla $x$ dolů.
                \item \texttt{ceil(x)} $\leftarrow$ zaokrouhlení čísla $x$ nahoru.
            \end{itemize}
            \item Všechny vypočtené hodnoty jsou v přesnosti \markblue{double}.
            \item Kompletní přehled (pro zájemce) \href{https://www.tutorialspoint.com/c_standard_library/math_h.htm}{\underline{zde}}
        \end{itemize}
    \end{frame}

    \begin{frame}[t]{Cykly}
        \begin{itemize}
            \item Často budeme potřebovat nějakou část kódu vykonat vícekrát.
            \item Je pochopitelně hloupost kopírovat stejný kód několikrát pod sebe. \emoji{slightly-smiling-face}
            \item K tomu se nám bude hodit tzv. cyklus (anglicky \emph{loop}).
            \item V jazyce C existují 3 typy:
            \begin{itemize}
                \item \texttt{While}
                \item \texttt{Do While}
                \item \texttt{For}
            \end{itemize}
        \end{itemize}
    \end{frame}

    \begin{frame}[t,fragile]{Cyklus \texttt{while}}
        \begin{itemize}
            \item Používáme, pokud dopředu neznáme počet opakování cyklu.
            \item Syntaxe:
            \begin{lstlisting}
while (podmínka) {
    // Kód programu
}
            \end{lstlisting}
            \item Odpovídá \textbf{iteraci s testem na začátku}.
        \end{itemize}
    \end{frame}

    \begin{frame}[fragile]{Příklad}
        \begin{lstlisting}
#include <stdio.h>
int main(void) {
    int n;
    scanf("%d", &n);
    while (n > 1) {
        print("%d", n);
        n = n / 2;
    }
    return 0;
}
        \end{lstlisting}
    \end{frame}

    \begin{frame}[t,fragile]{Příklad}
        I zde lze použít logické spojky (jako u \markblue{if}):
        \begin{lstlisting}
#include <stdio.h>
int main(void) {
    int x, y;
    scanf("%d %d", &x, &y);
    while (x > 1 || y < 1000) {
    printf("x = %d\ny = %d", x, y);
        x = x / 2;
        y = y * 2;
    }
    return 0;
}
        \end{lstlisting}
    \end{frame}

    \begin{frame}[t,fragile]{Cyklus \texttt{for}}
        \begin{itemize}
            \item Používáme, pokud naopak dopředu známe počet opakování.
            \item Syntaxe:
            \begin{lstlisting}
for (deklarace; podmínka; aktualizace) {
    // Kód
}
            \end{lstlisting}
        \end{itemize}
    \end{frame}

    \begin{frame}[t,fragile]{Cyklus \texttt{for}}
        \begin{itemize}
            \item Syntaxe:
            \begin{lstlisting}
for (deklarace; podmínka; aktualizace) {
    // Kód
}
            \end{lstlisting}
            \item Popis činnosti:
            \begin{itemize}
                \item provede se deklarace proměnné,
                \item zkontroluje se podmínka,
                \item pokud podmínka platí, provede se tělo cyklu,
                \item na vykonání těla cyklu se provede \textbf{aktualizace},
                \item znovu se zkontroluje podmínka a cyklus se opakuje,
                \item pokud podmínka neplatí, činnost cyklu končí.
            \end{itemize}
        \end{itemize}
    \end{frame}

    \begin{frame}[t,fragile]{Cyklus \texttt{for}}
        \begin{itemize}
            \item Typický příklad zápisu:
            \begin{lstlisting}
for (int i = 0; i < 10; i++) {
    // Kód
}
            \end{lstlisting}
            \item \texttt{i++} je ekvivalent zápisu \texttt{i = i + 1} (nejčastěji se používá právě u
            for cyklu).
        \end{itemize}
    \end{frame}

    \begin{frame}[t,fragile]{Příklad}
        Součet všech sudých čísel menších než $n$:
        \begin{lstlisting}
int n;
scanf("%d", &n);
int sum = 0;
for (int i = 0; i < n; i++) {
    if (i % 2 == 0) {
        sum += i;
    }
}
printf("%d", sum);
        \end{lstlisting}
    \end{frame}

    \begin{frame}[t,fragile]{Příklad}
        Nebo jinak:
        \begin{lstlisting}
int n;
scanf("%d", &n);
int sum = 0;
for (int i = 0; i < n; i += 2) {
    sum += i;
}
printf("%d", sum);
        \end{lstlisting}
    \end{frame}

    \begin{frame}[t,fragile]{Dodatek k desetinným číslům}
        \begin{itemize}
            \item \markorange{Nikdy neporovnáváme desetinná čísla na rovnost \xmark}
            \begin{lstlisting}
float x = 0.1;
x += 0.2;
if (x == 0.3) {
    // Kód
}
            \end{lstlisting}
            \item \markgreen{Lze pouze porovnávat max. odchylku \cmark}
            \begin{lstlisting}
if (fabs(x - 0.3) < 0.01) {
    // Kód
}
            \end{lstlisting}
        \end{itemize}
    \end{frame}

    \begin{frame}[t]{Dodatek k desetinným číslům}
        \begin{itemize}
            \item \textbf{Důvod:} při počítání s desetinnými čísly dochází k zaokrouhlovací chybě.
        \end{itemize}
    \end{frame}

    \begin{frame}{Otázky?}
        \begin{figure}
            \centering
            \includegraphics[scale=.4]{images/discussion_inverted.png}
        \end{figure}
    \end{frame}

\end{document}