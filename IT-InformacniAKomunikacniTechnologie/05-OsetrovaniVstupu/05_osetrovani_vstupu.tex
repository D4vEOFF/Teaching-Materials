\documentclass[14pt,aspectratio=169]{beamer}

% Assets
\usepackage[czech]{babel}				% Jazyk
%\usepackage[a-2u]{pdfx}					% Kopírování z pdfka
\usepackage{tikz}						% Schémata automatů
\usepackage[utf8]{inputenc}
\usepackage{textcomp}
\usepackage{hyperref}

\usefonttheme{serif}
\usepackage{lmodern}

\hypersetup{%
    pdfencoding=auto,
    pdfauthor={\insertauthor},
    pdftitle={\insertsubtitle}
}
\usepackage{csquotes}					% české uvozovky
\usepackage{enumerate}					% enumerate environment
\usepackage{indentfirst}
\usepackage{mathtools}
\usepackage{pifont}
\usepackage{soul}
\usepackage{xcolor}
\usepackage{graphicx}
\usepackage{amsmath}
\usepackage{emoji}
\usepackage{subfig}

\usepackage{listings}                   % Úryvky z kódu
% Beamer theme
\usetheme{Darmstadt}
\useoutertheme[subsection=false]{miniframes}
\definecolor{red}{rgb}{0.827, 0, 0}
\usecolortheme[named=red]{structure}
\setbeamertemplate{frame numbering}[fraction]
\setbeamertemplate{navigation symbols}{}
% Enumerate
%\setlist[enumerate]{topsep=0pt,itemsep=-1ex,partopsep=1ex,parsep=1ex,label=(\arabic*)}

\MakeOuterQuote{"}

% Colors
\definecolor{darkblue}{rgb}{.047,.047,.43}
\definecolor{darkgreen}{HTML}{0D7103}
\definecolor{lightgreen}{HTML}{68FF00}
\definecolor{darkred}{HTML}{AF0B0B}
\definecolor{lightred}{HTML}{FF5100}
\definecolor{orange}{HTML}{FFE000}

\newcommand{\markred}[1]{\textcolor{lightred}{#1}}
\newcommand{\markgreen}[1]{\textcolor{lightgreen}{#1}}
\newcommand{\markorange}[1]{\textcolor{orange}{#1}}

% Inline images
\newcommand{\inlineimgscale}{1.1}

% X and check mark
\newcommand{\cmark}{\ding{51}}
\newcommand{\xmark}{\ding{55}}

% Redefinions
\renewcommand{\implies}{\Rightarrow}
\renewcommand{\impliedby}{\Leftarrow}

% Title page
\subtitle{Informační a komunikační technologie}
\title{Ošetřování vstupu}
\author{David Weber}
\def\office{K13}
\def\email{weber3@spsejecna.cz}

\begin{document}

    % Itemize
    \setlist[itemize]{label=\textcolor{white}{\textbullet}}

    % Slides
    \begin{frame}
        \titlepage
    \end{frame}

    \begin{frame}[t,fragile]{Jak to bylo doposud\dots}
        \begin{itemize}
            \item Zatím jsme předpokládali inteligentního uživatele.
            \item Tzn. uživatel zadával pouze povolené hodnoty.
            \item \markred{V realitě uživatel však nemusí zadat požadovanou hodnotu!}
        \end{itemize}
    \end{frame}

    \begin{frame}[t,fragile]{Příklad na začátek}
        \begin{lstlisting}
int value;
scanf("%d", &value);

if (value == 0) {
    printf("1");
    return 0;
}
int fact = 1;
for (int i = 1; i <= value; i++) {
    fact *= i;
}

printf("%d", fact);
return 0;
        \end{lstlisting}
    \end{frame}

    \begin{frame}[t]{Příklad na začátek}
        \begin{itemize}
            \item Co bylo na tomto programu špatně?
            \item \markred{$\implies$ uživatel nemusí zadat \textbf{nezáporné celé číslo}!}
        \end{itemize}
    \end{frame}

    \begin{frame}[t,fragile]{Jak toto řešit?}
        \begin{itemize}
            \item Před každým programem je třeba si promyslet, které vstupy jsou přípustné.
            \item U předešlé úlohy by se hodilo ošetřit, zda je hodnota na vstupu přípustná.
            \item \markgreen{Můžeme např. na začátek programu přidat:}
            \begin{lstlisting}
if (n < 0) {
    printf("CHYBNY VSTUP!");
    return 0;
}
            \end{lstlisting}
        \end{itemize}
    \end{frame}

    \begin{frame}[t,fragile]{Problémy nekončí\dots}
        \begin{itemize}
            \item Uživatel nemusí zadat nic nebo např. znak.
            \item Toto již pomocí podmínek lze jen těžko odchytit. \emoji{crying-face}
            \item \markgreen{$\implies$ funkce \texttt{scanf} má však tzv. \textbf{návratovou hodnotu}!}
            \begin{lstlisting}
int value;
int paramCount = scanf("%d", &value);

printf("%d", paramCount);
            \end{lstlisting}
            \item V proměnné \texttt{paramCount} je uložen počet úspěšně načtených parametrů.
        \end{itemize}
    \end{frame}

    \begin{frame}[t,fragile]{Úprava programu}
        \begin{itemize}
            \item V případě program s faktoriálem uživatel zadává \textbf{jeden parametr}.
            \begin{lstlisting}
int value;
int paramCount = scanf("%d", &value);

if (paramCount < 1){
    printf("CHYBNY VSTUP!");
    return 0;
}
            \end{lstlisting}
        \end{itemize}
    \end{frame}

    \begin{frame}{Otázky?}
        \begin{figure}
            \centering
            \includegraphics[scale=.4]{images/discussion_inverted.png}
        \end{figure}
    \end{frame}

\end{document}