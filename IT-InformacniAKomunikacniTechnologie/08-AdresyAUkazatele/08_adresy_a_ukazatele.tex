\documentclass[14pt,aspectratio=169]{beamer}

% Assets
\usepackage[czech]{babel}				% Jazyk
%\usepackage[a-2u]{pdfx}					% Kopírování z pdfka
\usepackage{tikz}						% Schémata automatů
\usepackage[utf8]{inputenc}
\usepackage{textcomp}
\usepackage{hyperref}

\usefonttheme{serif}
\usepackage{lmodern}

\hypersetup{%
    pdfencoding=auto,
    pdfauthor={\insertauthor},
    pdftitle={\insertsubtitle}
}
\usepackage{csquotes}					% české uvozovky
\usepackage{enumerate}					% enumerate environment
\usepackage{indentfirst}
\usepackage{mathtools}
\usepackage{pifont}
\usepackage{soul}
\usepackage{xcolor}
\usepackage{graphicx}
\usepackage{amsmath}
\usepackage{emoji}
\usepackage{subfig}

\usepackage{listings}                   % Úryvky z kódu
% Beamer theme
\usetheme{Darmstadt}
\useoutertheme[subsection=false]{miniframes}
\definecolor{red}{rgb}{0.827, 0, 0}
\usecolortheme[named=red]{structure}
\setbeamertemplate{frame numbering}[fraction]
\setbeamertemplate{navigation symbols}{}
% Enumerate
%\setlist[enumerate]{topsep=0pt,itemsep=-1ex,partopsep=1ex,parsep=1ex,label=(\arabic*)}

\MakeOuterQuote{"}

% Colors
\definecolor{darkblue}{rgb}{.047,.047,.43}
\definecolor{darkgreen}{HTML}{0D7103}
\definecolor{lightgreen}{HTML}{68FF00}
\definecolor{darkred}{HTML}{AF0B0B}
\definecolor{lightred}{HTML}{FF5100}
\definecolor{orange}{HTML}{FFE000}

\newcommand{\markred}[1]{\textcolor{lightred}{#1}}
\newcommand{\markgreen}[1]{\textcolor{lightgreen}{#1}}
\newcommand{\markorange}[1]{\textcolor{orange}{#1}}

% Inline images
\newcommand{\inlineimgscale}{1.1}

% X and check mark
\newcommand{\cmark}{\ding{51}}
\newcommand{\xmark}{\ding{55}}

% Redefinions
\renewcommand{\implies}{\Rightarrow}
\renewcommand{\impliedby}{\Leftarrow}

% Title page
\title{Adresy a ukazatele v C}
\subtitle{Informační a komunikační technologie}
\author{David Weber}
\def\office{K13}
\def\email{weber3@spsejecna.cz}

\begin{document}

    % Itemize
    \setlist[itemize]{label=\textcolor{white}{\textbullet}}

    % Slides
    \begin{frame}
        \titlepage
    \end{frame}

    \begin{frame}[t]{Připomenutí}
        \begin{itemize}
            \item Jaký je rozdíl mezi \emph{adresou} a \emph{hodnotou} proměnné?
            \begin{itemize}
                \item \textbf{Hodnota} -- hodnota uložená v paměťové buňce proměnné
                \item \textbf{Adresa} -- číslo paměťové buňky proměnné
            \end{itemize}
            \item Pro výpis jsme používali znak \texttt{\&}.
        \end{itemize}
        \begin{figure}
            \centering
            \includegraphics{images/memory_variable.pdf}
        \end{figure}
        \begin{itemize}
            \item Předpona \texttt{0x} značí číslo zapsané \textbf{hexadecimálně} (v C, ale i jiných jazycích).
        \end{itemize}
    \end{frame}

    \begin{frame}[t,fragile]{Příklad na úvod I}
        Mějme program obsahující funkci pro prohození hodnot proměnných \texttt{u} a \texttt{v}.
        \begin{lstlisting}
void swap(int u, int v) {
    int temp = u;
    u = v;
    v = temp;
}
int main(void) {
    int a = 5;
    int b = 10;
    swap(a, b);
    printf("Hodnoty a, b: %d, %d", a, b);
    return 0;
}
        \end{lstlisting}
    \end{frame}

    \begin{frame}[t]{Příklad na úvod II}
        Jaký bude výstup předešlého programu?
        \begin{enumerate}[label=(\alph*)]
            \item \texttt{Hodnoty a, b: 5, 10}
            \only<1>{\item \texttt{Hodnoty a, b: 10, 5}}
            \only<2->{\item \markgreen{\texttt{Hodnoty a, b: 10, 5}}}
        \end{enumerate}
        \only<3->{\markred{$\implies$ nijak jsme si nepomohli \emoji{crying-face}}}
    \end{frame}

    \begin{frame}[t]{V čem je problém?}
        \begin{itemize}
            \item Parametry funkci předáváme tzv. \textbf{hodnotou}.
            \begin{itemize}
                \item Hodnoty proměnných \texttt{a} a \texttt{b} jsou zkopírovány a při volání funkce \texttt{swap} jsou nově \emph{na zásobníku} deklarovány proměnné \texttt{u} ~a \texttt{v}.
                \item \markred{$\implies$ funkce v konečném důsledku prohodí hodnoty proměnných \texttt{u} a \texttt{v}, nikoliv \texttt{a} a \texttt{b}}
            \end{itemize}
            \item \markgreen{$\implies$ mohli bychom vyřešit předáním ``odkazů'' na původní proměnné.}
        \end{itemize}
    \end{frame}

    \begin{frame}[t,fragile]{Ukazatel (pointer)}
        \begin{itemize}
            \item \textbf{Datový typ} uchovávající adresu v paměti \textbf{určitého datového typu} (existují i generické ukazatele, ale ty nebudeme řešit \emoji{slightly-smiling-face}).
            \item Při deklaraci je třeba uvést datový typ (tím kompilátoru říkáme, jak se má interpretovat místo v paměti, kam ukazuje ), jehož adresu ukazatel uchovává, a znak \texttt{*} (pro odlišení od deklarace standardní proměnné).
            \item Jedná se vždy o \textbf{kladné celé číslo} (formátová specifikace \texttt{\%p}).
            \begin{lstlisting}
int main(void) {
    int x = 50;
    int *px = &x;
    printf("Hodnota x: %d\nAdresa x: %p", x, px);
    return 0;
}
            \end{lstlisting}
        \end{itemize}
    \end{frame}

    \begin{frame}[t,fragile]{Dereference}
        \begin{itemize}
            \item Ekvivalentně lze deklaraci ukazatele zápisem \texttt{int* p} (mezera mezi \texttt{*} a \texttt{p}).
            \item Co když máme adresu, ale ne samotnou proměnnou? $\implies$ \markgreen{využijeme tzv. \textbf{operátor dereference} \texttt{*}.}
            \begin{itemize}
                \item Umožňuje nám odkázat se přímo na \textbf{hodnotu} v paměťové buňce ležící na \textbf{adrese ukazatele}.
            \end{itemize}
            \begin{lstlisting}
int main(void) {
    int i = 50;
    int *p = &i;
    *p = 100;           // same as i = 100;
    printf("%d", i);    // prints out "100"
}
            \end{lstlisting}
        \end{itemize}
    \end{frame}

    \begin{frame}[t,fragile]{Otázka}
        Jak bychom měli upravit původní program, aby fungoval?
        \begin{lstlisting}
void swap(int u, int v) {
    int temp = u;
    u = v;
    v = temp;
}
int main(void) {
    int a = 5;
    int b = 10;
    swap(a, b);
    printf("Hodnoty a, b: %d, %d", a, b);
    return 0;
}
        \end{lstlisting}
    \end{frame}

    \begin{frame}{Otázky?}
        \begin{figure}
            \centering
            \includegraphics[scale=.4]{images/discussion_inverted.png}
        \end{figure}
    \end{frame}

\end{document}