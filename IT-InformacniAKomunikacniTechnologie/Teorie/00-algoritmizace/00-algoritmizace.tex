\documentclass[11pt,aspectratio=169]{beamer}

% Assets
\usepackage[czech]{babel}				% Jazyk
%\usepackage[a-2u]{pdfx}					% Kopírování z pdfka
\usepackage{tikz}						% Schémata automatů
\usepackage[utf8]{inputenc}
\usepackage{textcomp}
\usepackage{hyperref}

\usefonttheme{serif}
\usepackage{lmodern}

\hypersetup{%
    pdfencoding=auto,
    pdfauthor={\insertauthor},
    pdftitle={\insertsubtitle}
}
\usepackage{csquotes}					% české uvozovky
\usepackage{enumerate}					% enumerate environment
\usepackage{indentfirst}
\usepackage{mathtools}
\usepackage{pifont}
\usepackage{soul}
\usepackage{xcolor}
\usepackage{graphicx}
\usepackage{amsmath}
\usepackage{emoji}
\usepackage{subfig}

\usepackage{listings}                   % Úryvky z kódu
% Enumerate
%\setlist[enumerate]{topsep=0pt,itemsep=-1ex,partopsep=1ex,parsep=1ex,label=(\arabic*)}

\MakeOuterQuote{"}

% Colors
\definecolor{darkblue}{rgb}{.047,.047,.43}
\definecolor{darkgreen}{HTML}{0D7103}
\definecolor{lightgreen}{HTML}{68FF00}
\definecolor{darkred}{HTML}{AF0B0B}
\definecolor{lightred}{HTML}{FF5100}
\definecolor{orange}{HTML}{FFE000}

\newcommand{\markred}[1]{\textcolor{lightred}{#1}}
\newcommand{\markgreen}[1]{\textcolor{lightgreen}{#1}}
\newcommand{\markorange}[1]{\textcolor{orange}{#1}}

% Inline images
\newcommand{\inlineimgscale}{1.1}

% X and check mark
\newcommand{\cmark}{\ding{51}}
\newcommand{\xmark}{\ding{55}}

% Redefinions
\renewcommand{\implies}{\Rightarrow}
\renewcommand{\impliedby}{\Leftarrow}
% Beamer theme
\usetheme{Darmstadt}
\useoutertheme[subsection=false]{miniframes}
\definecolor{red}{rgb}{0.827, 0, 0}
\usecolortheme[named=red]{structure}
\setbeamertemplate{frame numbering}[fraction]
\setbeamertemplate{navigation symbols}{}

% Title page
\title{Algoritmizace}
\author{David Weber}
\institute{SPŠE Ječná}
\date{\today}

\begin{document}

    % Itemize
    \setlist[itemize]{label=\textcolor{black}{\textbullet}}

    % Slides
    \begin{frame}
        \titlepage
    \end{frame}

    \section{Tvorba programu}
    \begin{frame}[t]{Algoritmizace}
        \only<2->{Chci něco spočítat, jak na to?}
        \only<3->{
        \begin{itemize}
            \only<3->{\item Rozmyslím si postup výpočtu}
            \only<4->{\item Provedu výpočet podle vymyšleného postupu}
        \end{itemize}
        }
        \only<5->{Potřebuji nutně rozumnět postupu?}
        \only<6->{\begin{center}
            \markred{$\implies$ Nepotřebuji, důležitá je jeho správnost!}
        \end{center}}
    \end{frame}
    \begin{frame}{Algoritmus}
        \only<2>{\begin{block}{Co je to algoritmus?}
            Přesný návod či postup, kterým lze vyřešit daný typ úlohy.
        \end{block}}
    \end{frame}
    \begin{frame}[t]{Algoritmus}
        \only<2->{V užším slova smyslu se algoritmem rozumí takové postupy, které mají určité vlastnosti.}
        \only<3->{
            \begin{block}{Vlastnosti algoritmu}
                \begin{itemize}
                    \item \textbf{Elementárnost (diskrétnost).} Algoritmus se skládá z konečného počtu jednoduchých (elementárních) kroků.
                    \only<4->{\item \textbf{Konečnost (finitnost).} Algoritmus musí skončit v konečném počtu kroků.}
                    \only<5->{\item \textbf{Obecnost (hromadnost).} Algoritmus neřeší jeden konkrétní problém, ale obecnou třídu obdobných problémů (např. neřeší jen "kolik je $2\cdot 2$", ale obecně součin libovolné dvojice čísel $a\cdot b$).}
                    \only<6->{\item \textbf{Determinovanost.} Po každém kroku lze jednoznačně určit, který následuje.}
                    \only<7->{\item \textbf{Správnost.} Algoritmus řeší danou úlohu, tj. pro přípustná data vydá správný výsledek a nesprávná vstupní data zamítne.}
                \end{itemize}
            \end{block}
        }
    \end{frame}

\end{document}