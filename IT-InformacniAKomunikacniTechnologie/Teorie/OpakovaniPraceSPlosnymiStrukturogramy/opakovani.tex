\documentclass{beamer}

\usepackage[czech]{babel}				% Jazyk
%\usepackage[a-2u]{pdfx}					% Kopírování z pdfka
\usepackage{tikz}						% Schémata automatů
\usepackage[utf8]{inputenc}
\usepackage{textcomp}
\usepackage{hyperref}

\usefonttheme{serif}
\usepackage{lmodern}

\hypersetup{%
    pdfencoding=auto,
    pdfauthor={\insertauthor},
    pdftitle={\insertsubtitle}
}
\usepackage{csquotes}					% české uvozovky
\usepackage{enumerate}					% enumerate environment
\usepackage{indentfirst}
\usepackage{mathtools}
\usepackage{pifont}
\usepackage{soul}
\usepackage{xcolor}
\usepackage{graphicx}
\usepackage{amsmath}
\usepackage{emoji}
\usepackage{subfig}

\usepackage{listings}                   % Úryvky z kódu

% Assets
% Enumerate
%\setlist[enumerate]{topsep=0pt,itemsep=-1ex,partopsep=1ex,parsep=1ex,label=(\arabic*)}

\MakeOuterQuote{"}

% Colors
\definecolor{darkblue}{rgb}{.047,.047,.43}
\definecolor{darkgreen}{HTML}{0D7103}
\definecolor{lightgreen}{HTML}{68FF00}
\definecolor{darkred}{HTML}{AF0B0B}
\definecolor{lightred}{HTML}{FF5100}
\definecolor{orange}{HTML}{FFE000}

\newcommand{\markred}[1]{\textcolor{lightred}{#1}}
\newcommand{\markgreen}[1]{\textcolor{lightgreen}{#1}}
\newcommand{\markorange}[1]{\textcolor{orange}{#1}}

% Inline images
\newcommand{\inlineimgscale}{1.1}

% X and check mark
\newcommand{\cmark}{\ding{51}}
\newcommand{\xmark}{\ding{55}}

% Redefinions
\renewcommand{\implies}{\Rightarrow}
\renewcommand{\impliedby}{\Leftarrow}

% Theme
\usetheme{Boadilla}
\setbeamertemplate{frame numbering}[fraction]
\usecolortheme[named=darkblue]{structure}
\setbeamertemplate{navigation symbols}{}

% Title page
\subtitle{Práce s plošnými strukturogramy}
\author{David Weber}
\date{\today}

\begin{document}

    \begin{frame}[t]
        \titlepage
    \end{frame}

    \begin{frame}[t]{Nejdříve zopáčko...}
        \only<2->{Co vypíše program níže?}
        \only<3->{\begin{figure}
            \includegraphics[scale=0.8]{images/priklad1_opakovani.pdf}
        \end{figure}}
    \end{frame}

    \begin{frame}[t]{Nejdříve zopáčko...}
        \only<2->{Výstup bude XXXXXXXXX.}
        \only<3->{\begin{center}
            \[\underbrace{\overbrace{\markgreen{XXX}}^{i=1}\;\overbrace{\markred{XXX}}^{i=2}\;\overbrace{\markorange{XXX}}^{i=3}}_{\text{9-krát}}.\]
        \end{center}}
        \only<4->{Pro každé opakování vnější iterace se vnitřní iterace provede třikrát $\implies$ $3\cdot 3=9$.}
    \end{frame}

    \begin{frame}[t]{Upravená varianta}
        \only<2->{Zkusme program trochu upravit. Jak se změní výstup programu, bude-li vnitřní iterace \textbf{s testem na konci}?}
        \only<3->{\begin{figure}
            \includegraphics[scale=0.8]{images/priklad2_opakovani.pdf}
        \end{figure}}
    \end{frame}

    \begin{frame}{Upravená varianta}
        Kód vnitřní iterace se provede opět třikrát $\implies$ \markgreen{výstup bude stejný. :-)}
    \end{frame}

    \begin{frame}[t]{Obecný vstup}
        \only<2->{Zkusme program zobecnit. Nechceme, aby počet vypsaných symbolů byl vždy pevný\dots}
        \only<3->{\begin{center}
            $\implies$ \markgreen{PŘIDÁME UŽIVATELSKÝ VSTUP}
        \end{center}}
        \only<4->{Program se na začátku zeptá uživatele na jisté číslo $n$, s nímž pak dále pracuje.}
    \end{frame}

    \begin{frame}{Obecný vstup}
        \begin{figure}
            \includegraphics[scale=0.8]{images/priklad3_opakovani.pdf}
        \end{figure}
    \end{frame}

    \begin{frame}[t]
        \only<2->{Počet vypsaných symbolů \textit{X} již není pevný (je závislý na hodnotě $n$).\\}
        \only<3->{Vnější iterace se nyní provede $n$-krát a pro každé její opakování se vnitřní iterace provede právě třikrát.}
        \only<4->{\begin{center}
            \[\underbrace{\overbrace{\markgreen{XXX}}^{i=1}\;\overbrace{\markred{XXX}}^{i=2}\;\overbrace{\markorange{XXX}}^{i=3}\;\overbrace{\markgreen{XXX}}^{i=4}\;\cdots\;\overbrace{\markorange{XXX}}^{i=n}}_{\text{??}}\]
        \end{center}}
        \only<5->{$\implies$ $n\cdot 3=\markgreen{3n}$ symbolů X.}
    \end{frame}

    \begin{frame}[t]{Samostatná práce}
        \only<2->{Kolik symbolů X vypíše tento program?}
        \only<3->{\begin{figure}
            \includegraphics[scale=0.8]{images/priklad4_opakovani.pdf}
        \end{figure}}
    \end{frame}

    \begin{frame}
        \only<2->{\begin{center}
            \[\overbrace{\markgreen{XXX\dots X}}^{n}\;\overbrace{\markred{XXX\dots X}}^{n}\;\overbrace{\markorange{XXX\dots X}}^{n}\;\cdots\;\overbrace{\markgreen{XXX\dots X}}^{n}\]
        \end{center}}
        \only<3->{Pro každé opakování vnější iterace se vnitřní iterace se zopakuje $n$-krát} \only<3->{$\implies$ správná odpověď je $\markgreen{n\cdot n=n^2}$ symbolů X. :-)}
    \end{frame}

\end{document}