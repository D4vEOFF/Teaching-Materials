\documentclass[14pt,aspectratio=169]{beamer}

% Assets
\usepackage[czech]{babel}				% Jazyk
%\usepackage[a-2u]{pdfx}					% Kopírování z pdfka
\usepackage{tikz}						% Schémata automatů
\usepackage[utf8]{inputenc}
\usepackage{textcomp}
\usepackage{hyperref}

\usefonttheme{serif}
\usepackage{lmodern}

\hypersetup{%
    pdfencoding=auto,
    pdfauthor={\insertauthor},
    pdftitle={\insertsubtitle}
}
\usepackage{csquotes}					% české uvozovky
\usepackage{enumerate}					% enumerate environment
\usepackage{indentfirst}
\usepackage{mathtools}
\usepackage{pifont}
\usepackage{soul}
\usepackage{xcolor}
\usepackage{graphicx}
\usepackage{amsmath}
\usepackage{emoji}
\usepackage{subfig}

\usepackage{listings}                   % Úryvky z kódu
% Beamer theme
\usetheme{Darmstadt}
\useoutertheme[subsection=false]{miniframes}
\definecolor{red}{rgb}{0.827, 0, 0}
\usecolortheme[named=red]{structure}
\setbeamertemplate{frame numbering}[fraction]
\setbeamertemplate{navigation symbols}{}
% Enumerate
%\setlist[enumerate]{topsep=0pt,itemsep=-1ex,partopsep=1ex,parsep=1ex,label=(\arabic*)}

\MakeOuterQuote{"}

% Colors
\definecolor{darkblue}{rgb}{.047,.047,.43}
\definecolor{darkgreen}{HTML}{0D7103}
\definecolor{lightgreen}{HTML}{68FF00}
\definecolor{darkred}{HTML}{AF0B0B}
\definecolor{lightred}{HTML}{FF5100}
\definecolor{orange}{HTML}{FFE000}

\newcommand{\markred}[1]{\textcolor{lightred}{#1}}
\newcommand{\markgreen}[1]{\textcolor{lightgreen}{#1}}
\newcommand{\markorange}[1]{\textcolor{orange}{#1}}

% Inline images
\newcommand{\inlineimgscale}{1.1}

% X and check mark
\newcommand{\cmark}{\ding{51}}
\newcommand{\xmark}{\ding{55}}

% Redefinions
\renewcommand{\implies}{\Rightarrow}
\renewcommand{\impliedby}{\Leftarrow}

% Title page
\title{Informační a komunikační technologie}
\subtitle{Funkce v C}
\author{David Weber}
\def\office{K13}
\def\email{weber3@spsejecna.cz}

\begin{document}

    % Itemize
    \setlist[itemize]{label=\textcolor{white}{\textbullet}}

    % Slides
    \begin{frame}
        \titlepage
    \end{frame}

    \begin{frame}[t,fragile]{Příklad na úvod}
        \begin{lstlisting}
// Print array
for (int i = 0; i < size; i++) {
    printf("%d ", arr[i]);
}
printf("\n");
// Square array elements
for (int i = 0; i < size; i++) {
    arr[i] *= arr[i];
}
// Print array
for (int i = 0; i < size; i++) {
    printf("%d ", arr[i]);
}
        \end{lstlisting}
    \end{frame}

    \begin{frame}[t,fragile]{V čem je problém?}
        \begin{itemize}
            \item Kód je funkční, ale část pro výpis pole je zde uvedena \textbf{dvakrát}.
            \begin{lstlisting}
for (int i = 0; i < size; i++) {
    printf("%d ", arr[i]);
}
            \end{lstlisting}
            \item \markred{Opakující kód je nepraktický} \xmark
            \begin{itemize}
                \item budeme-li chtít změnit nějakou jeho část, musíme změnu provést všude
            \end{itemize}
            \item \markgreen{Použijeme funkci} \cmark
        \end{itemize}
    \end{frame}

    \begin{frame}[t]{Co je to funkce?}
        \begin{itemize}
            \item Obecně se jedná o část programu, kterou je možné opakovaně ``vyvolat'' v různých místech programu.
            \item Motivace pro použití:
            \begin{itemize}
                \item odstranění opakování kódu v programu,
                \item rozklad složitých problémů na jednodušší,
                \item znovupoužití v jiných programech, např. formou knihoven (to až ve druháku \emoji{shushing-face}).
            \end{itemize}
        \end{itemize}
    \end{frame}

    \begin{frame}[t,fragile]{Struktura funkce}
        \begin{lstlisting}
navratovy_typ jmeno(parametry) {
    <implementace>
}
        \end{lstlisting}
        \begin{itemize}
            \item U funkce je třeba specifikovat:
            \begin{itemize}
                \item návratový datový typ,
                \item jméno (identifikátor),
                \item parametry,
                \item tělo (implementace).
            \end{itemize}
            \item Funkce uvádíme mimo tělo funkce \texttt{main}.
        \end{itemize}
    \end{frame}

    \begin{frame}[t,fragile]{Funkce bez parametrů}
        \begin{itemize}
            \item Nejjednodušší typ funkce.
            \item Klíčové slovo \markblue{\texttt{void}} (prázdný datový typ), do závorek \texttt{(\markblue{void})} nebo nechat prázdné, tj. \texttt{()}
            \begin{lstlisting}
void greet(void) {
    printf("Hello World!");
}
            \end{lstlisting}
            \item \textbf{Důležité:} funkci musíme uvést nad \texttt{main}!
            \item Samotná definice funkce nic nedělá $\implies$ \markorange{je třeba ji tzv. \textbf{zavolat}.}
        \end{itemize}
    \end{frame}

    \begin{frame}[t,fragile]{Volání funkce}
        \begin{itemize}
            \item Je třeba specifikovat, kde v programu se má daná funkce provést.
            \begin{lstlisting}
void greet(void) {
    printf("Hello World!");
}
            
int main(void) {

    greet();    // Prints out "Hello World!"

    return 0;
}
            \end{lstlisting}
        \end{itemize}
    \end{frame}

    \begin{frame}[t,fragile]{Funkce s parametry}
        \begin{itemize}
            \item Často budeme chtít činnost funkce zobecnit (výstup funkce bude na něčem záviset).
            \item $\implies$ k tomu použijeme tzv. \textbf{parametry}.
            \item Do kulatých závorek \texttt{()} uvádíme výčet parametrů (jejich datový typ a název), které funkce přijímá.
            \begin{lstlisting}
void fce(<datovy_typ> var1, <datovy_typ> var2, ...) {
    ...
}
            \end{lstlisting}
        \end{itemize}
    \end{frame}

    \begin{frame}[t,fragile]{Příklad I (funkce s parametrem)}
        \begin{lstlisting}
void square(float x) {
    printf("Druha mocnina x je %g", x*x);
}

int main(void) {

    float input;
    scanf("%f", &input);

    square(input);
    
    return 0;
}
        \end{lstlisting}
    \end{frame}

    \begin{frame}[t,fragile]{Příklad II (funkce s více parametry)}
        Funkce pro výpočet aritmetického průměru celých čísel $a,\,b\in\mathbb{Z}$:
        \begin{lstlisting}
void average(int a, int b) {
    float avg = (a + b) / 2;
    printf("%g", avg);
}

int main(void) {
    float input1, input2;
    scanf("%d %d", &input1, &input2);

    average(input1, input2);
    
    return 0;
}
        \end{lstlisting}
    \end{frame}

    \begin{frame}{Úloha I}
        Předělejte program pro výpis faktoriálu, tj.
        \begin{equation*}
            n!=n(n-1)(n-2)\cdots 2\cdot 1,
        \end{equation*}
        do funkce. (Viz domácí úkol \emoji{slightly-smiling-face})
    \end{frame}

    \begin{frame}[t,fragile]{Funkce s návratovou hodnotou}
        \begin{itemize}
            \item \textbf{Problém:} s hodnotou, kterou funkce vypočte, nemůžeme dále pracovat (pouze ji na konci vypisujeme).
            \item Např. u funkcí \texttt{pow}, \texttt{sqrt}, \dots (viz knihovna \textbf{math}) jsme mohli s vypočtenou hodnotou dále počítat
            \begin{lstlisting}
float hypotenuse = sqrt(pow(x, 2) + pow(y, 2));
printf("%f", hypotenuse);
            \end{lstlisting}
            \item $\implies$ \markgreen{využijeme tzv. \textbf{návratovou hodnotu}.}
        \end{itemize}
    \end{frame}

    \begin{frame}{Otázky?}
        \begin{figure}
            \centering
            \includegraphics[scale=.4]{images/discussion_inverted.png}
        \end{figure}
    \end{frame}

\end{document}