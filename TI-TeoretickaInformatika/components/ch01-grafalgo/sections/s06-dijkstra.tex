\section{Dijkstrův algoritmus}\label{sec:dijkstra}

Již jsme si ukázali, že v \emph{neohodnoceném grafu} $G=(V,E)$ umíme relativně jednoduše najít délku nejkratší cesty do libovolného vrcholu z nějakého výchozího vrcholu $v_0\in V$. Obvykle však hrany nemusí mít stejnou váhu (cenu). Např. cesty (vrcholy) mezi městy (hrany) mohou být v různém stavu a některé jsou tak lepší než jiné.
\importantbox{Zde ji narážíme na problém, neboť obecně platí, že když nalezneme vrchol přes dvojici různých hran, nemusí být nalezené cesty stejně dlouhé. Na obrázku \ref{fig:ohod_graf} si lze všimnout, že cesta $(v_0,v_1,v_2)$ je kratší než $(v_0,v_2)$, přestože má více hran.}
\begin{figure}[h]
    \centering
    \includegraphics[scale=\graphimgsize]{components/images/ch01_ohod_graf.pdf}
    \caption{Příklad ohodnoceného grafu.}
    \label{fig:ohod_graf}
\end{figure}
Nabízí se varianta převést ohodnocený graf na neohodnocený tak, že rozdělíme hranu na takový počet (neohodnocených) hran, kolik činí její původní váha. V čistě teoretické rovině se jedná o funkční řešení, neboť na nově vzniklý graf již lze aplikovat např. BFS, které jsme popsali v sekci \ref{sec:bfs}. Ne každý graf však musí mít "malé" váhy hran. Pokud bychom vzali např. graf, kde váhy hran jsou v řádech tisíců, bude pro BFS potřeba provést zbytečně mnoho práce, neboť pro zpracování jedné ohodnocené hrany bude třeba provést tisíce iterací.

Jedním z nejznámějších algoritmů v tomto ohledu je tzv. \emph{Dijkstrův algoritmus}, který popsal v roce 1958 holandský informatik \emph{Edsger Wybe Dijkstra}.\footnote{Jedná se o holandské jméno, čteme "dajkstra".} Podobně jako u BFS a DFS, i zde budeme postupně \emph{otevírat} a \emph{uzavírat} vrcholy, přičemž každý uzavřeme \emph{nejvýše jednou}, avšak je třeba mít na paměti důležitou věc.
\importantbox{První nalezená cesta do libovolného vrcholu již nemusí být nutně nejkratší. Cestu do daného vrcholu bude třeba postupně vylepšovat.}
Zaměřme se na grafy, jejichž váhy hran jsou nezáporné. Záporně ohodnocené hrany mohou způsobovat problémy, neboť mezi určitou dvojicí vrcholů by již nemusela nutně existovat nejkratší cesta. Např. v obrázku \ref{fig:zapor_ohod_graf} si lze všimnout, že mezi vrcholy $v_0$ a $v_4$ neexistuje nejratší cesta konečné délky, neboť cyklus $(v_1, v_3, v_2, v_1)$ lze projít libovolněkrát, přičemž každým průchodem se zkrátí o $3$. Odpověď na délku nejrakší cesty mezi těmito vrcholy by tak musela být $-\infty$.
\begin{figure}[h]
    \centering
    \includegraphics[scale=\graphimgsize]{components/images/ch01_zapor_ohod_graf.pdf}
    \caption{Graf, kde mezi $v_0$ a $v_4$ neexistuje (konečná) nejkratší cesta.}
    \label{fig:zapor_ohod_graf}
\end{figure}
Podobným grafům se tak chceme vyhnout, neboť by naše úvahy pak již nemusely fungovat (cestu mezi vrcholy bychom mohli potenciálně do nekonečně vylepšovat a algoritmus by se tak nikdy nezastavil).

V dalších odstavcích budeme váhu hrany vedoucí z vrcholu $u$ do vrcholu $v$ značit $\ell(u,v)$.
\begin{pseudo}{Dijkstra}[Nezáporně ohodnocený graf $G=(V,E)$ a počáteční vrchol $v_0\in V$.][Seznam vzdáleností $D$.]
    \Comment{Inicializace}
    \begin{For}{všechny vrcholy $v\in V$}
        $stav(v)\gets$ \textit{nenalezený}\\
        $D(v)\gets\infty$\\
    \end{For}
    $stav(v_0)\gets$ \textit{otevřený}\\
    $D(v_0)\gets 0$\\
    \begin{While}{existují otevřené vrcholy}
        \Comment{Kandidát na nejkraší cestu do sousedních vrcholů}
        $v\gets$ otevřený vrchol s nejmenším $D$\\

        \begin{For}{všechny sousedy $w\in V$ vrcholu $v$}
            \Comment{Našli jsme lepší cestu}
            \begin{If}{$D(v)+\ell(v,w)<D(w)$}
                $D(w)=\gets D(v)+\ell(v,w)$\\
                $stav(w)\gets$ \textit{otevřený}
            \end{If}
        \end{For}
        $stav(v)\gets$ \textit{uzavřený}
    \end{While}
\end{pseudo}
\begin{figure}[h]
    \centering
    \includegraphics[scale=\graphimgsize]{components/images/ch01_dijkstra_kratsi_cesta.pdf}
    \caption{Znázornění situace v průběhu Dijkstrova algoritmu.}
    \label{fig:dijkstra_kratsi_cesta}
\end{figure}

\subsubsection{Rozbor Dijkstrova algoritmu}

Nejprve je dobré si uvědomit, proč algoritmus vlastně funguje. Kdykoliv uzavíráme vrchol (tzn. vybereme vrchol s nejmenší hodnotou $D(v)$), pak se spoléháme na to, že $D(v)$ opravdu odpovídá v daný moment délce nejkraší cesty z $v_0$ do $v$. Ale proč to tak musí být?

Zkusme uvážit opačnou situaci, tzn. kdyby nastalo, že jsme vybrali nějaký vrchol $v$, jehož $D(v)$ neodpovídá délce nejkratší cesty. Pak by musel existovat nějaký sousední vrchol $u$ vrcholu $v$, přes nějž vede nejkratší cesta, jejíž délka je $D(u)+\ell(u,v)$ (viz obrázek ). To znamená, že $D(v)>D(u)+\ell(u,v)$. Jenže v takovou chvíli by si algoritmus nutně musel vybrat vrchol $u$ místo vrcholu $v$, protože
\[D(v)>D(u)+\ell(u,v)\stackrel{\texttt{*}}{>}D(u).\]
V kroku označeném \texttt{*} vycházíme právě z toho, že váhy hran jsou nezáporné, tj. $\ell(u,v)>0$ (proto tento argument funguje). Tedy stručně řečeno\footnote{Nebo spíš napsáno?}, pokud $D(v)$ neodpovídá délce nejkratší cesty, pak musí existovat jiný vrchol $u$ s nižším $D(u)$.
\begin{figure}[h]
    \centering
    \includegraphics[scale=\graphimgsize]{components/images/ch01_dijkstra_vyber_vrcholu.pdf}
    \caption{Situace při výběru vrcholu $v$, kdy $D(v)$ neodepovídá délce nejkratší cesty.}
    \label{fig:dijkstra_vyber_vrcholu}
\end{figure}

\subsubsection{Časová složitost Dijkstrova algoritmu}

