\section{Binární halda}\label{sec:halda}

Uveďme motivační příklad na úvod. Mějme seznam čísel, z něhož chceme (pokud možno co nejrychleji) vybrat minimální/maximální hodnotu. Pro maximum bychom sestavit následující jednoduchý algoritmus.
\begin{pseudo}{Max}[Seznam čísel $x_1,x_2,\dots,x_n$][Maximální hodnota seznamu \textit{max}]
    $max\gets x_1$\\
    \begin{For}{$i=1,2,\dots,n$}
        \begin{If}{$x_i>max$}
            $max\gets x_i$
        \end{If}
    \end{For}
\end{pseudo}
Časová složitost bude zjevně $\bigO{n}$, neboť algoritmus prochází všech $n$ prvků. Takovou úlohu lze však řešit rychleji, pokud si prvky vhodně uspořádáme. K tomu můžeme použít tzv. \emph{haldu}. 
\begin{definition}[Minimová binární halda]
    Minimová binární halda je datová struktura tvaru binárního stromu, kde v každém vrcholu je uložena \emph{právě jedna} hodnota (tzv. \emph{klíč}, pro vrchol $v$ budeme značit jeho klíč $k(v)$) a navíc platí:
    \begin{enumerate}[label=(\roman*)]
        \item každá hladina je plně obsazena, kromě poslední
        \item a je-li $v$ libovolný vrchol a $s$ jeho syn, pak $k(v)\leqslant k(v)$.
    \end{enumerate}
\end{definition}

%%% Doplnit obrázky (modely/anti-modely)

