\section{Prohledávání do hloubky}\label{sec:dfs}
Na podobném přístupu, jako BFS, je založeno tzv. \emph{prohledávání do hloubky} (anglicky \emph{depth-first search}, zkráceně DFS). Vrcholy však tentokrát budeme zpracovávat rekurzivně. Pokaždé, když budeme otevírat nový vrchol, se rekurzivně zavoláme všechny jeho sousední vrcholy, u nichž opakujeme stejnou proceduru. Po prozkoumání všech sousedů daný vrchol uzavřeme. Stejně jako BFS si tak budeme uchovávat pole \emph{stavů} pro jednotlivé vrcholy.
\begin{pseudo}{DFS}[Graf $G=(V,E)$ a počáteční vrchol $v_0\in V$.]
    \Comment{Inicializace}
    \begin{For}{každý vrchol $v\in V$}
        $stav(v)\gets$ \textit{nenalezený}
    \end{For}
    Zavolej DFS2($v_0$)\\

    \Comment{Rekurzivní volání na vrcholy}
    \begin{Function}{DFS2}[vrchol $v$]
        $stav(v)\gets$ \textit{otevřený}\\
        \begin{For}{každý sousední vrchol $w$ vrcholu $v$}
            \begin{If}{$stav(w)=$ \textit{nenalezený}}
                Zavolej DFS2($w$)
            \end{If}
        \end{For}
        $stav(v)\gets$ \textit{uzavřený}
    \end{Function}
\end{pseudo}
\begin{theorem}[Složitost DFS]\label{thm:dfs_slozitost}
    Algoritmus DFS doběhne v čase $\bigO{n+m}$ a spotřebuje paměť $\bigO{n+m}$.
\end{theorem}
\begin{proof}
    Algoritmus DFS se oproti BFS liší pouze v pořadí, v jakém pořadí dosažitelné vrcholy zpracovává, to však nemá na časovou složitost žádný vliv. Argument pro její odvození je tak stejný jako u BFS, viz věta \ref{thm:bfs_slozitost} v minulé sekci.

    V paměti si musíme uchovávat reprezentaci grafu, to zabere $\bigO{n+m}$ paměti (opět např. pomocí seznamu sousedů) a máme seznam $stav$ s $n$ prvky (vrcholy). Zároveň si při volání \textsc{DFS2} musíme na zásobník rekurze ukládat jednotlivé \emph{aktivační záznamy}. Protože vrcholů je v grafu $n$, pak na zásobníku rekurze bude v jednu chvíli maximálně $n$ záznamů, tj. $\bigO{n}$. Celkově spotřebujeme $\bigO{n+m}+\bigO{n}+\bigO{n}=\bigO{n+m}$ paměti.
\end{proof}
\begin{remark}
    \begin{itemize}
        \item Je dobré si uvědomit, že byť algoritmy BFS a DFS mají stejnou časovou a prostorovou složitost, nelze zcela rovnocenně použít na stejné typy úloh. Např. BFS se hodí pro hledání nejkratší cesty v neohodnoceném grafu (pro ohodnocené grafy se používá např. Dijkstrův algoritmus), kdežto DFS se více hodí na prohledávání stavového prostoru, neboť typicky nezabere tolik paměti.
        \item Cesta, kterou se DFS dostane do libovolného vrcholu $v$, nemusí být nutně nejkratší (silně závisí na pořadí, v jakém procházíme sousedy jednotlivých vrcholů). 
    \end{itemize}
\end{remark}