\chapter{Algoritmicky těžké problémy}\label{chap:tezke-problemy}

Mnoho problémů lze v informatice řešit pomocí polynomiálních algoritmů, tzn. běžících v čase $\bigO{n^k}$ pro nějaké prvné $k$. Např. problém řazení prvků v poli jsme schopni řešit pomocí různých algoritmů, např. \emph{BubbleSortem}, jehož časová složitost je v nejhorším případě $\bigO{n^2}$, kde $n$ je počet prvků pole. Podobně hledání nejkratší cesty v grafu dokážeme řešit polynomiálně např. Dijkstrovým algoritmem, který i v případě implementace pomocí pole běží v čase $\bigO{n^2+m}$, kde $n$ je počet vrcholů a $m$ počet hran\footnote{Může se zdát matoucí, že zde figurují dvě proměnné $n$ a $m$ místo jedné, ale počet hran je omezený (nejvýše mohou být každé dva vrcholy spojeny hranou) a to výrazem $n(n+1)/2$, což je polynom. Tedy všechny prezentované grafové algoritmy běží v polynomiálním čase}. S jistou rezervou lze říci, že polynomiální algoritmy jsou prakticky dobře použitelné.

Bohužel ne vždy je situace takto příznivá. Lze totiž narazit na problémy, na které není známý polynomiální algoritmus a o některých dokonce s jistotou víme, že je v polynomiálním čase řešit nelze. Dobrým příkladem jsou v tomto ohledu např. \emph{Hanojské věže}\footnote{Pro zájemce podrobnější vysvětlení: \url{https://en.wikipedia.org/wiki/Tower_of_Hanoi}}, kde nejlepší algoritmus je exponenciální, tj. $\bigO{2^n}$, neboť je vždy potřeba exponenciálně mnoho kroků k vyřešení.

Nás budou zajímat ty nejdůležitější problémy z této oblasti, protože mezi nimi lze nalézt zajímavé vztahy, a posléze si uděláme menší ochutnávku z problematiky \P{} a \NP.

\section{Problém SAT}\label{sec:sat}

Jako první začneme zdánlivě možná jednoduchým problémem, který však na poli informatiky hraje dosti důležitou roli. Předpokládejme, že máme zadanou nějakou logickou formuli $\varphi$ o libovolném počtu logických proměnných, označme např. $x_1,x_2,\dots,x_n$. Naší otázkou je, jestli je možné hodnoty proměnných $x_1,x_2,\dots,x_n$ nastavit tak (tj. dosadit hodnoty 0 a 1), aby byla formule splněna\footnote{Může se na první pohled zdát, že se jedná o dosti teoretickou záležitost, nicméně později uvidíme, že mnoho jiných problémů má se SAT silnou spojitost.}, tj. $\varphi(\dots)=1$. Takovému ohodnocení budeme říkat \emph{splňující}. Z toho vychází i název samotného problému SAT (z angl. \emph{satisfiability}, neboli splnitelnost).