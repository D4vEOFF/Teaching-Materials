\section{Problém SAT}\label{sec:sat}

Jako první začneme zdánlivě možná jednoduchým problémem, který však na poli informatiky hraje dosti důležitou roli. Předpokládejme, že máme zadanou nějakou logickou formuli $\varphi$ o libovolném počtu logických proměnných, označme např. $x_1,x_2,\dots,x_n$. Naší otázkou je, jestli je možné hodnoty proměnných $x_1,x_2,\dots,x_n$ nastavit tak (tj. dosadit hodnoty 0 a 1), aby byla formule splněna\footnote{Může se na první pohled zdát, že se jedná o dosti teoretickou záležitost, nicméně později uvidíme, že mnoho jiných problémů má se SAT silnou spojitost.}, tj. $\varphi(\dots)=1$. Takovému ohodnocení budeme říkat \emph{splňující}. Z toho vychází i název samotného problému SAT (z angl. \emph{satisfiability}, neboli splnitelnost).