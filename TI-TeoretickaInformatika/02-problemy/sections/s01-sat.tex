\section{Problém SAT}\label{sec:sat}

Jako první začneme zdánlivě možná jednoduchým problémem, který však na poli informatiky hraje dosti důležitou roli. Předpokládejme, že máme zadanou nějakou logickou formuli $\varphi$ o libovolném počtu logických proměnných, označme např. $x_1,x_2,\dots,x_n$. Naší otázkou je, jestli je možné hodnoty proměnných $x_1,x_2,\dots,x_n$ nastavit tak (tj. dosadit hodnoty 0 a 1), aby byla formule splněna\footnote{Může se na první pohled zdát, že se jedná o dosti teoretickou záležitost, nicméně později uvidíme, že mnoho jiných problémů má se SAT silnou spojitost.}, tj. $\varphi(\dots)=1$. Takovému ohodnocení budeme říkat \emph{splňující}. Z toho vychází i název samotného problému SAT (z angl. \emph{satisfiability}, neboli splnitelnost).

\problem{Splnitelnost logické formule (SAT)}{Logická formule $\varphi$.}{1, pokud je $\varphi$ splnitelná, jinak 0.}

Nejdříve si zopakujeme logické operace a spojky. Mezi ně řadíme $\neg, \land, \lor, \implies, \iff$.
\begin{itemize}
    \item \textbf{Negace $\neg x$.} \emph{Převrací logickou hodnotu $x$.}
    \item \textbf{Konjunkce $a\land b$.} \emph{Pravdivá, jsou-li obě hodnoty operandů $a,b$ pravdivé.}
    \item \textbf{Disjunkce $a\lor b$.} \emph{Pravdivá, je-li alespoň jedna z hodnot operandů $a,b$ pravdivá.}
    \item \textbf{Implikace $a\implies b$.} \emph{Je-li předpoklad $a$ splněn, je splněn i závěr $b$. Tj. implikace je nepravdivá, pokud platí $a$ a neplatí $b$.}
    \item \textbf{Ekvivalence $a\iff b$.} \emph{$a$ platí právě tehdy, když platí $b$. Tj. ekvivalence je pravdivá, jsou-li hodnoty operandů $a,b$ současně 0 nebo 1.}
\end{itemize}
\begin{table}[h]\label{table:logicke_operace}
    \centering
    \begin{tabular}{|c|c|c|c|c|c|c|}
    \hline
    $a$ & $b$ & $\neg a$ & $a\land b$ & $a\lor b$ & $a\implies b$ & $a\iff b$ \\ \hline
    0   & 0   & 1        & 0          & 0         & 1             & 1         \\ \hline
    0   & 1   & 1        & 0          & 1         & 1             & 0         \\ \hline
    1   & 0   & 0        & 0          & 1         & 0             & 0         \\ \hline
    1   & 1   & 0        & 1          & 1         & 1             & 1         \\ \hline
    \end{tabular}
\end{table}
Zde konstatujme, že ve všech příkladech bude mít negace $\neg$ vždy \emph{nejvyšší prioritu} oproti ostatním operacím. Prioritu ostatních operací budeme stanovovat pomocí závorek. 
\begin{example}\label{ex:sat_formule}
    \begin{itemize}
        \item $\varphi(x)=x\land\neg x$ \dots \textbf{ Není splnitelná}, pro $x=0$ i $x=1$ je $\varphi=0$.
        \item $\psi(x)=x\lor\neg x$ \dots \textbf{ Je splnitelná}, a to jak pro $x=0$, tak $x=1$.
        \item $\chi(a,b,c)=\big((a\land b)\lor \neg c\big)\implies (\neg a\land\neg c)$ \dots \textbf{Je splnitelná}, např. pro $a=0$, $b=0$ a $c=0$.
    \end{itemize}
\end{example}
V příkladu \ref{ex:sat_formule} výše se jedná o poměrně jednoduché instance, kdy jsme schopni vidět hned, zda je formule splnitelná. Pokud bychom však uvážili nějakou složitější formuli, problém se již značně zkomplikuje.
\importantbox{Obecně pro formuli o $n$ proměnných je třeba prozkoumat řádově $2^n$ možných ohodnocení. Naivní algoritmus zkoušející všechny možnosti je tak \emph{exponenciální}.}
Pro SAT není dodnes známý žádný algoritmus, který by jej uměl řešit pro libovolnou logickou formuli v polynomiálním čase.

\subsection{Konjunktivní normální forma}\label{subsec:cnf}

Zkusíme se podívat na formule ve speciálním tvaru, a to tzv. \emph{konjunktivní normální formě}, neboli zkráceně \emph{CNF}.\footnote{Z angl. \emph{conjunctive normal form}.}

Každá formule skládá z tzv. \textbf{klauzulí} obsahujících tzv. \textbf{literály}, což je buď \emph{proměnná, nebo její negace}.
\begin{itemize}
    \item Klauzule jsou ve formuli odděleny \emph{logickou spojkou $\land$}
    \item a v každé klauzuli jsou literály odděleny \emph{logickou spojkou $\lor$}.
\end{itemize}
\[\varphi(\dots)=\underbrace{(\cdots\lor\cdots\lor\cdots)}_{\text{klauzule}}\land(\cdots\lor \overbrace{x}^{\text{literál}}\lor\cdots)\land(\cdots\lor\overbrace{\neg x}^{\text{literál}}\lor\cdots)\land(\cdots\lor \underbrace{y}_{\text{literál}}\lor\cdots)\land\cdots\]
\begin{example}\label{ex:cnf_formule}
    \begin{itemize}
        \item \(\varphi(x,y,z)=\neg x\land (y\lor z)\) \dots \textbf{Je v CNF.} Formule $\varphi$ obsahuje klauzule $x$ (klauzule o jednom literálu) a $y\lor z$.
        \item \(\psi(a,b)=a\land b\) \dots \textbf{Je v CNF.}
        \item \(\chi(a,b,c,d)=a\land \big(b\lor (c\land d)\big)\) \dots \textbf{Není v CNF}, protože výraz $c\land d$ není literál.
    \end{itemize}
    Poslední formuli lze však převést do CNF: $a\land (b\lor c)\land (b\lor d)$.
\end{example}
Z příkladu \ref{ex:cnf_formule} se tak nabízí otázka, zda je možné \emph{libovolnou formuli} převést do CNF. Existuje vůbec pro každou formuli taková ekvivalentní\footnote{Libovolné formule $\varphi$ a $\psi$ jsou ekvivaletní, pokud pro stejné ohodnocení proměnných platí $\varphi(\dots)=\psi(\dots)$.} formule? Ale ano, v logice se i dokazuje, že každé formuli existuje existuje ekvivalentní formule v CNF.
\begin{theorem}
    Pro každou logickou formuli $\psi$ existuje ekvivalentní formule $\psi^\prime$ v CNF.
\end{theorem}
Toto tvrzení lze, pochopitelně, dokázat, ale zde se tím zabývat nebudeme a zkrátka jej přijmeme jako fakt. Podstatná věc, která z toho plyne, je, že pokud je nějaká formule $\psi$ splnitelná, pak je splnitelná i jí ekvivalentní formule $\psi^\prime$ v CNF a naopak pokud $\psi$ není splnitelná, není splnitelná ani $\psi^\prime$. Symbolicky
\[\psi\,\text{je splnitelná}\iff\psi^\prime\,\text{je splnitelná}.\]

\subsection{Převod do CNF pomocí tabulky}

Podívejme se na způsob, jak lze převést libovolnou logickou formuli do CNF.

Máme obecnou formuli $\varphi$ s proměnnými $x_1,x_2,\dots,x_n$, pro kterou budeme konstruovat ekvivalentní formuli $\varphi^\prime$ v CNF. Vycházíme z tabulky pravdivostníh hodnot, přičemž pro každé \textbf{nepravdivé} ohodnocení vytvoříme \emph{klauzuli} s $n$ literály, kde obecně $i$-tý literál je
\begin{itemize}
    \item $x_i$, pokud v daném ohodnocení je $x_i=0$,
    \item jinak $\neg x_i$ (tj. když $x_i=1$).
\end{itemize}
Podívejme se na úvod na příklad \ref{ex:prevod_cnf_1} níže.
\begin{example}\label{ex:prevod_cnf_1}
    Máme formuli $\psi(x,y,z)=(x\implies y)\implies z$, pro kterou chceme najít ekvivalentní formuli $\psi^\prime$ v CNF. Nejprve si tedy sestavíme tabulku pravdivostních hodnot.
    \begin{table}[h]
        \centering
        \begin{tabular}{|c|c|c|c|c|}
        \hline
        $x$ & $y$ & $z$ & $x\implies y$ & $(x\implies y)\implies z$ \\ \hline
        0   & 0   & 0   & 1             & 0                         \\ \hline
        0   & 0   & 1   & 1             & 1                         \\ \hline
        0   & 1   & 0   & 1             & 0                         \\ \hline
        0   & 1   & 1   & 1             & 1                         \\ \hline
        1   & 0   & 0   & 0             & 1                         \\ \hline
        1   & 0   & 1   & 0             & 1                         \\ \hline
        1   & 1   & 0   & 1             & 0                         \\ \hline
        1   & 1   & 1   & 1             & 1                         \\ \hline
        \end{tabular}
    \end{table}
    Z tabulky můžeme vidět, že formule $\varphi$ je nepravdivá pro
    \begin{itemize}
        \item $x=0,y=0,z=0$,
        \item $x=0,y=1,z=0$
        \item a $x=1,y=1,z=0$.
    \end{itemize}
    Tedy celkově sestavíme 3 klauzule:
    \begin{itemize}
        \item \((\overbrace{x}^{x=0}\lor \underbrace{y}_{y=0}\lor \overbrace{z}^{z=0})\),
        \item \((x\lor\underbrace{\neg y}_{y=1}\lor z)\),
        \item \((\neg x\lor\neg y\lor z)\).
    \end{itemize}
    Výsledná formule v CNF bude mít tvar:
    \[\psi^\prime(x,y,z)=(x\lor y\lor z)\land (x\lor \neg y\lor z)\land (\neg x\lor \neg y\lor z).\]
    Čtenář se sám může přesvědčit, že formule $\psi$ a $\psi^\prime$ pro libovolné ohodnocení mají stejnou pravdivostní hodnotu.
\end{example}
Nabízí se otázka: \emph{"proč tento postup funguje?"} Pokud nějaká formule v CNF není splněna pro určité ohodnocení, pak to nutně znamená, že \emph{alespoň jedna z klauzulí není splněna}. Každá klauzule však obsahuje literály spojené logickou spojkou $\lor$ a tedy existuje pro ni jen jediné ohodnocení daných literálů, tak, aby nebyla splněna. Z tohoto proncipu vychází ona konstrukce. Vybereme si ta ohodnocení původní formule, která ji nesplňují a sestrojíme takovou klauzuli, aby právě pro dané ohodnocení nebyla splňena.
\begin{example}
    \(\chi(x_1,x_2,x_3,x_4)=(x_1\iff x_2)\lor \neg(x_3\iff x_4)\).
    \begin{table}[h]
        \centering
        \begin{tabular}{|c|c|c|c|c|c|c|}
        \hline
        $x_1$ & $x_2$ & $x_3$ & $x_4$ & $x_1\iff x_2$ & $\neg(x_3\iff x_4)$ & $(x_1\iff x_2)\lor\neg(x_3\iff x_4)$ \\ \hline
        0     & 0     & 0     & 0     & 1             & 0                                                 & 1                                    \\ \hline
        0     & 0     & 0     & 1     & 1             & 1                                                 & 1                                    \\ \hline
        0     & 0     & 1     & 0     & 1             & 1                                                 & 1                                    \\ \hline
        0     & 0     & 1     & 1     & 1             & 0                                                 & 1                                    \\ \hline
        0     & 1     & 0     & 0     & 0             & 0                                                 & 0                                    \\ \hline
        0     & 1     & 0     & 1     & 0             & 1                                                 & 1                                    \\ \hline
        0     & 1     & 1     & 0     & 0             & 1                                                 & 1                                    \\ \hline
        0     & 1     & 1     & 1     & 0             & 0                                                 & 0                                    \\ \hline
        1     & 0     & 0     & 0     & 0             & 0                                                 & 0                                    \\ \hline
        1     & 0     & 0     & 1     & 0             & 1                                                 & 1                                    \\ \hline
        1     & 0     & 1     & 0     & 0             & 1                                                 & 1                                    \\ \hline
        1     & 0     & 1     & 1     & 0             & 0                                                 & 0                                    \\ \hline
        1     & 1     & 0     & 0     & 1             & 0                                                 & 1                                    \\ \hline
        1     & 1     & 0     & 1     & 1             & 1                                                 & 1                                    \\ \hline
        1     & 1     & 1     & 0     & 1             & 1                                                 & 1                                    \\ \hline
        1     & 1     & 1     & 1     & 1             & 0                                                 & 1                                    \\ \hline
        \end{tabular}
    \end{table}
    \[\chi^\prime(x_1,x_2,x_3,x_4)=(x_1\lor x_2\lor \neg x_3\lor x_4)\land(x_1\lor\neg x_2\lor\neg x_3\lor\neg x_4)\land(\neg x_1\lor x_2\lor x_3\lor x_4)\land(\neg x_1\lor x_2\lor \neg x_3\lor\neg x_4).\]
    \importantbox{\textbf{Problém:} Generování tabulky (resp. procházení všech možných ohodnocení) nás přivádí zpět k původnímu problému, a to sice \emph{exponenciální časové složitosti}.}
    V tomto případě však nejsme schopni postup učinit více efektivní, neboť ve skutečnosti se může stát, že pro obecnou logickou formuli se může jí ekvivalentní formule v CNF až exponenciálně prodloužit.\footnote{Stačí např. vzít formuli, která je \emph{nesplnitelná}, tj. pro všechna možná ohodnocení platí $\psi(\dots)=0$. Celkově by tak ekvivalentní formule měla až $2^n$ klauzulí.}
\end{example}