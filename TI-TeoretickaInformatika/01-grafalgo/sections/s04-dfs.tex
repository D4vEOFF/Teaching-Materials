\section{Prohledávání do hloubky}\label{sec:dfs}
Na podobném přístupu, jako BFS, je založeno tzv. \emph{prohledávání do hloubky} (anglicky \emph{depth-first search}, zkráceně DFS). Vrcholy však tentokrát budeme zpracovávat rekurzivně. Pokaždé, když budeme otevírat nový vrchol, se rekurzivně zavoláme všechny jeho sousední vrcholy, u nichž opakujeme stejnou proceduru. Po prozkoumání všech sousedů daný vrchol uzavřeme. Stejně jako BFS si tak budeme uchovávat pole \emph{stavů} pro jednotlivé vrcholy.
\begin{algorithm}\label{alg:dfs}
    \caption{DFS (prohledávání do hloubky)}
    \KwIn{Graf $G=(V,E)$ a počáteční vrchol $v_0\in V$}
    \tcp{Inicializace}
    \ForEach{\textup{vrchol $v\in v$}}{
        $stav(v)\gets$ \textit{nenalezený}
    }
    Zavolej DFS2($v_0$)\\
    \tcp{Rekurzivní volání na sousední vrcholy}
    \Function{\textup{DFS2(vrchol $v\in V$)}}{
        $stav(v)\gets$ \textit{otevřený}\\
        \ForEach{\textup{sousední vrchol $w$ vrcholu $v$}}{
            \If{$stav(w)=$ \textit{nenalezený}}{
                Zavolej DFS2($w$)
            }
        }
    }
\end{algorithm}

\begin{theorem}[Složitost DFS]\label{thm:dfs_slozitost}
    Algoritmus DFS doběhne v čase $\bigO{n+m}$ a spotřebuje paměť $\bigO{n+m}$.
\end{theorem}
\begin{proof}
    Algoritmus DFS se oproti BFS liší pouze v pořadí, v jakém pořadí dosažitelné vrcholy zpracovává, to však nemá na časovou složitost žádný vliv. Argument pro její odvození je tak stejný jako u BFS, viz věta \ref{thm:bfs_slozitost} v minulé sekci.

    V paměti si musíme uchovávat reprezentaci grafu, to zabere $\bigO{n+m}$ paměti (opět např. pomocí seznamu sousedů) a máme seznam $stav$ s $n$ prvky (vrcholy). Zároveň si při volání \textsc{DFS2} musíme na zásobník rekurze ukládat jednotlivé \emph{aktivační záznamy}. Protože vrcholů je v grafu $n$, pak na zásobníku rekurze bude v jednu chvíli maximálně $n$ záznamů, tj. $\bigO{n}$. Celkově spotřebujeme $\bigO{n+m}+\bigO{n}+\bigO{n}=\bigO{n+m}$ paměti.
\end{proof}
\notebox{Je dobré si uvědomit, že byť algoritmy BFS a DFS mají stejnou časovou a prostorovou složitost, nelze zcela rovnocenně použít na stejné typy úloh. Např. BFS se hodí pro hledání nejkratší cesty \emph{v neohodnoceném grafu} (pro ohodnocené grafy se používá např. Dijkstrův algoritmus, viz sekce \ref{sec:dijkstra}), kdežto DFS se více hodí na prohledávání stavového prostoru, neboť typicky nezabere tolik paměti.}
\warningbox{Cesta, kterou se DFS dostane do libovolného vrcholu $v$, nemusí být nutně nejkratší (silně závisí na pořadí, v jakém procházíme sousedy jednotlivých vrcholů).}